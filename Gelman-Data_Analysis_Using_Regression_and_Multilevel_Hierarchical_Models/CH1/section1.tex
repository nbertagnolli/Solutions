\subsection*{Problem 1}
  Think about what it means to scale this data.  We have a set of data $X$ and this data has a mean $\mu_x$ and standard deviation $\sigma$.  We want to linearly transform the data which means:
  \begin{align*}
    Y &= a + bX \\
    \mu_y &= a + b\mu_x \\
    \sigma_y &= |b| \sigma_x
  \end{align*}

Now to answer (a) and (b) we can just solve these equations.  Let's solve for standard deviation.

\begin{align*}
15 &= |b| 10\\
1.5 &= |b|
\end{align*}

This means that $b = \pm 1.5$ and we can now solve for the mean to get that $a = 47.5$ or $a = 152.5$.  We can transform it with either the first set of values or the second.  You don't want to use the second because that will reverse the ordering in the data.  

  
 \subsection*{Problem 2}

 \subsubsection*{(a)}
 
 Doing this in R is easy:
 \begin{verbatim}
 girls <- c(.4777, .4875, .4874, .4859, .4754, .4864,.4813,.4787,.4895,
           .4797,.4876,.4859,.4857,.4907, .5010,.4903,.4860,.4911,.4871,
           .4725,.4822,.4870,.4823,.4973)

num_births <- 3903
std <- sd(girls)
avg <- mean(girls)
expected_std <- sqrt(avg * (1 - avg) / num_births)
 \end{verbatim}
 
 We get that std$= .0064$ and expected\_std$=.008$
 
 Let's quickly prove this formula for the expected standard error of a proportion.
 \begin{proof}
   We are trying to prove that $SE(X) = \sqrt{\frac{p(1-p)}{n}}$  for binary variables $x_i$ which take on only values 0 and 1.
   
   Begin by assuming that we have a population with $m$ instances where $x=1$ and $n-m$ instances where $x=0$ We know that the standard deviation of a population is:
   \begin{align*}
     \sigma &= \sqrt{\frac{\sum(x_i - \bar{x})^2}{n}} 
   \end{align*}
   Let's just concern ourselves with the summation portion of this equation and see if we can massage it into the desirable form.   
   \begin{align*}
     \sum(x_i - \bar{x})^2 &= \sum (x_i - \bar{x})(x_i-\bar{x}) \\
     &= \sum x_i^2 - 2x_i\bar{x} + \bar{x}^2 \\
     &= \sum_{x_i = 0}  \bar{x}^2  + \sum_{x_i = 1} 1 - 2\bar{x} + \bar{x}^2 \\
     &= (n-m) \bar{x}^2 + m - 2m\bar{x} + m\bar{x}^2\\
     &= (n-m) \frac{m^2}{n^2} + m - 2m\frac{m}{n} + m\frac{m^2}{n^2}&& \bar{x} = \frac{m}{n}\\
     &= m + \frac{m^2}{n} - \frac{m^3}{n^2} - \frac{2m^2}{n} + \frac{m^3}{n^2}\\
     &= m - \frac{m^2}{n} \\
     &= m\left(1-\frac{m}{n}\right)\\
     &= np(1-p) && p=m/n
   \end{align*}
   So we can finish by substituting this back into our original equation!
   \begin{align*}
    \sigma  &=  \sqrt{\frac{np(1-p)}{n}}  \\
    &= \sqrt{p(1-p)}
   \end{align*}
   
   Since the sample proportion is a mean the standard error is calculated like normal yielding
   \begin{align*}
   SE(X) &= \frac{\sigma_x}{\sqrt{n}}\\
   &= \sqrt{\frac{p(1-p)}{n}}
   \end{align*}
 \end{proof}
 
 \subsubsection*{(b)}
No they are not significant.  If we run a $\chi^2$ test we get $\chi^2 = .0019$ which is no where near what we need for a significant result.
 
 
 \subsection*{Problem 3}
 \begin{verbatim}
cols <- 20
rows <- 1000
x <- replicate(cols, runif(rows))
data <- apply(a, 1, sum)
avg <- mean(data)
std <- sd(data)
hist(data, density=20)
curve(dnorm(x, mean=avg, sd=std), 
      col="darkblue", lwd=2, add=TRUE, yaxt="n")
 \end{verbatim}
 
 The plot is:
  \begin{figure}[!ht]
  \centering
    \includegraphics[width=0.5\textwidth]{CH1/1_3_hist.png}
\end{figure}

\subsection*{Problem 4}
\begin{verbatim}
i = 1
height_diff = seq(1000) * 0
while (i <= 1000) {
  men_height <- rnorm(100, 69.1, 2)
  women_height <- rnorm(100, 63.7, 2.7)
  height_diff[i] = mean(men_height) - mean(women_height)
  i = i + 1
}

avg <- mean(height_diff)
std <- sd(height_diff)
hist(height_diff, density=20, freq=FALSE)
curve(dnorm(x, mean=avg, sd=std), 
      col="darkblue", lwd=2, add=TRUE, yaxt="n")
\end{verbatim}

The mean of the difference is 5.05 and the standard deviation is .79
The plot is:

  \begin{figure}[!ht]
  \centering
    \includegraphics[width=0.5\textwidth]{CH1/1_4_hist.png}
\end{figure}

\subsection*{Problem 5}
Expectation is linear so:
\begin{align*}
\mathbb{E}[X+Y] &= \mathbb{E}[X] + \mathbb{E}[Y]
\end{align*}

As to standard deviation:

\begin{align*}
\text{Var}[X+Y] &= \text{Var}[X] + \text{Var}[Y] + \text{Cov}[X, Y]\\
&= \text{Var}[X] + \text{Var}[Y] + \text{Corr}[X,Y]\sigma_x\sigma_y
\end{align*}
 
 
 
  



    
  
