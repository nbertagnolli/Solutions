\Large\textbf{Problem 17}

The gamma function is defined as:

\begin{align*}
\Gamma(x) &= \int_0^\infty u^{x-1}e^{-u}du
\end{align*}

We use integration by parts with the following substitutions

\begin{align*}
u &= u^{x-1}  & v &= -e^{-u}\\
du &= (x-1)u^{x-2}du  & dv &=  e^{-u}du
\end{align*}

By integration by parts we have:

\begin{align*}
\Gamma(x) &= -u^{x-1}e^{-u}|_0^\infty + (x-1)\int_0^\infty u^{x-2}e^{-u}du\\
&=  (x-1)\int_0^\infty u^{x-2}e^{-u}du\\
&= (x-1)\Gamma(x-1)\\
&= z \Gamma(z) && z = x+1
\end{align*}

Now see that $\Gamma(1) = 1$

\begin{align*}
\Gamma(1) &= \int_0^\infty u^{0}e^{-u}du\\
&=  \int_0^\infty e^{-u}du\\
&= -e^{-u}|_0^\infty\\
&= 1
\end{align*}

To show that for all $x \in \mathbb{Z}^+$ $\Gamma(x + 1) = x!$ we use an inductive argument.  Our base case is $\Gamma(1) = 1$  which we have already done.  Now let's assume that our inductive hypothesis is true and $\Gamma(x + 1) = x!$.  Then $\Gamma(x + 2) = (x+1)\Gamma(x+1) = (x + 1)x!$ and we are done.


\Large\textbf{Problem 18}

We want to solve for $S_D$ the surface area of a $D$ dimensional sphere. 

\[\prod_{i=1}^D \int_{-\infty}^\infty e^{-x_i^2} = S_D \int_0^\infty e^{-r^2}r^{D-1}dr\]

Let's solve the left hand side first.  From problem 7 above we know that

\[ \int_{-\infty}^\infty e^{-\frac{1}{2\sigma^2}x^2} = \sqrt{2\pi \sigma^2}\]

By setting $\sigma^2 = \frac{1}{2}$ we get that

\begin{align*}
\prod_{i=1}^D \int_{-\infty}^\infty e^{-x^2} &= \prod_{i=1}^D\sqrt{\pi}\\
&= \pi^{\frac{D}{2}}
\end{align*}

Now let's look at the right hand side of the equation.  We can tease this function into the Gamma function by making the substitution $u = r^2$  which implies that $dr = \frac{1}{2}u^{-1/2}$.  With this insight we see that:

\begin{align*}
\int_0^\infty e^{-r^2}r^{D-1}dr &= \frac{1}{2}\int_0^\infty e^{-u}u^{\frac{D}{2}-1}du\\
&= \frac{1}{2} \Gamma(D/2)
\end{align*}

Now just use simple algebra:

\begin{align*}
\pi^{\frac{D}{2}} &= S_D \frac{1}{2} \Gamma(D/2)\\
S_D &= \frac{2\pi^{D/2}}{\Gamma(D/2)}
\end{align*}.

Next integrate the surface area with respect to the radius in order to get the equation for volume:

\begin{align*}
V_D &= S_D\int_0^1r^{D-1}dr\\
&= S_D \left.\frac{r}{D}\right|_0^1\\
&= \frac{S_D}{D}
\end{align*}

Now to derive some common known volumes.  For D =2:

\begin{align*}
\frac{2\pi^{D/2}}{D\Gamma(D/2)} &= \frac{2\pi}{2\Gamma(1)} \\
&= \pi
\end{align*}

For D=3:

\begin{align*}
\frac{2\pi^{D/2}}{D\Gamma(D/2)} &= \frac{2\pi^{3/2}}{3\Gamma(3/2)} \\
&= \frac{2\pi^{3/2}}{3\frac{\sqrt{\pi}}{2}} \\
&= \frac{4\pi}{3}
\end{align*}


\Large\textbf{Problem 19}

From problem 18 we know what the volume of a hypersphere is.  The volume of a D-dimensional cube is simply $l^D$ where $l$ is the length of a side.  In our case this is $2a$ so the ratio is:

\begin{align*}
\frac{\text{Volume Sphere}}{\text{Volume Cube}} &= \frac{2\pi^{D/2}a^D}{(2a)^DD\Gamma(D/2)}\\
&= \frac{\pi^{D/2}}{2^{D-1}D\Gamma(D/2)}
\end{align*}

To show that the ratio goes to zero I'm going to eschew Stirling's formula because there is a much easier way to show this....  Begin by noticing that the above equation is always positive.  Therefore we know that $0 \leq \frac{\pi^{D/2}}{2^{D-1}D\Gamma(D/2)}$ $\forall D \in \mathbb{Z}$.  Now let's replace the gamma function by the factorial function because they are identical for integer operands.  I will also drop the $D$ and $2^{D-1}$ from the bottom which increases the total value of the expression giving me:

\begin{align*}
0 &\leq \frac{\pi^{D/2}}{2^{D-1}D\Gamma(D/2)} \leq \frac{\pi^{D/2}}{(D/2)!}\\
0 &\leq \frac{\pi^{x}}{x2^{2x}\Gamma(x)} \leq \frac{\pi^{x}}{x!}\\
\end{align*}

Now we just need to show that $\lim_{x\rightarrow \infty}\frac{\pi^x}{x!} = 0$.  Which is easy!  Notice that:

\begin{align*}
0 &< \frac{\pi^x}{x!} \\
\frac{\pi^x}{x!} &= \frac{\pi	}{1}\cdot \frac{\pi}{2} \cdot \frac{\pi}{3} \cdot ... \cdot \frac{\pi}{x}\\
&< \frac{\pi	}{1}\cdot \frac{\pi}{2} \cdot \frac{\pi}{3} \cdot \frac{\pi}{4} \cdot \frac{\pi}{4} \cdot ... \cdot \frac{\pi}{4} \\
&= \frac{\pi}{6}\left(\frac{\pi}{4}\right)^{n-3}
\end{align*}

Now we know that:

\begin{align*}
\lim_{x\rightarrow \infty} \left(\frac{\pi}{4}\right)^{n-3} &= 0
\end{align*}

Therefore by the squeeze theorem so does $\frac{\pi^{D/2}}{2^{D-1}D\Gamma(D/2)}$ and we are done.


To show that the ratio of the distance from the center to a corner to the distance from the center to a side is $\sqrt{D}$ start by centering the cube at the origin.  Then we know that one corner will be at $(a, a,...,a)$ and the distance to that corner will be $a\sqrt{D}$ in euclidean space.  Now look at the distance to a side.  A side will be at the point $(a, 0, 0, ....)$ which will be distance $a$ from the origin in Euclidean space so the ratio is $\sqrt{D}$ 




